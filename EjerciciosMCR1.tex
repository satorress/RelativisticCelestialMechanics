
\documentclass[english]{article}
\usepackage[latin9]{inputenc}
\usepackage{amstext}
\usepackage{babel}
\begin{document}

\title{Ejercicios Mecanica Celeste Newtoniana}

\maketitle
1. El cometa Halley se mueve en una orbita con eccentricidad $e=0.967$
y el perihelio está ubicado a una distancia de $88'513.920\mbox{ km}$
del Sol. Encuentre el periodo orbital y las velocidades del cometa
en el perihelio y en el afhelio.\\


2. Un cometa se observa por primera vez a una distancia de $k$ unidades
astronómicas (UA) del Sol y se observa que está viajando con una velocidad
de $s$ veces la velocidad orbital media de la Tierra. Muestre que
la orbita del cometa es hiperbólica, parabólica o elíptica, dependiendo
de si la cantidad $s^{2}k$ es mayor, igual o menor que $2$, respectivamente.\\


3. Cuantos dias le toma al radio vector Sol-Tierra para rotar $90\text{º}$,
comenzando en el perihelio? Cuanto le tarda comenzando en el afelio?
(Considere la eccentricidad de la orbita terrestre como $e=0.01673$
y el periodo orbital como $\tau=365.24\mbox{ dias}$).\\


4. Considere un asteroide que orbita el Sol. Demuestre que para un
valor de energía fijo, la orbita que minimiza el momento angular orbital
es una trayectoria circular.

5. Considere un cometa en una orbita eliptica alrededor del Sol. Defina
las coordenadas cartesianas $\left(x,y\right)$ en el plano orbital,
tales que el punto $\left(0,0\right)$ corresponde al Sol y el eje
$x$ apunta en forma paralela al eje mayor orbital en dirección del
perihelio. Demuestre que

\begin{eqnarray*}
x & = & a\left(\cos E-e\right)\\
y & = & a\sqrt{1-e^{2}}\sin E
\end{eqnarray*}
con $a$ el semieje mayor, $e$ la eccentricidad y $E$ la anomalia
Eccentrica.
\end{document}
